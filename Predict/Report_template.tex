\documentclass[a4paper]{article}
\usepackage[top=1in,bottom=1in,left=1in,right=1in]{geometry}
\usepackage[utf8]{inputenc}
\usepackage{amsmath}
\usepackage{amssymb}
\usepackage{setspace}
\usepackage{color}
\usepackage{hyperref}
\usepackage{placeins}
\usepackage{mathtools}
\usepackage{graphicx}
\usepackage{subcaption}

\title{STAT 153 Project}
\author{Group: Jared Fisher, Johnny Hong, Dan Soriano}
\date{\today}

\begin{document}

\maketitle

\noindent\textbf{Executive Summary:}

\textit{In the beginning of your report, in a short summary, summarize in a few (about 1-4) sentences which data set you analyzed, which model you used to make predictions, and briefly describe how your predictions address the scenario presented in the README file of your dataset. }

LaTex is tool used to create professional-looking documents. Below, we shall show
how to insert equations, figures, and tables, as well as introduce
the sections expected of the report.
%
This template is not intended to give you a step-by-step instruction on how to do the analysis.
%
% and the % sign denotes a comment that is not processed by LaTeX
%
It only helps you with Latex and shows you some general aspect you should take care about in your report. 
%
However, the sections which are introduced in this template should appear exactly like this in your report, that is 1. Exploratory Data Analysis, 2. Model of time, 3. ARIMA Model Selection, 4. Results.
%
For more information about LaTex, here are a few good reference to get started
\begin{itemize}
\item \url{https://www.overleaf.com/learn/latex/Learn_LaTeX_in_30_minutes}
\item \url{https://www.latex-project.org/about/}
\item \url{https://en.wikibooks.org/wiki/LaTeX/Introduction}
\end{itemize}


\section{Exploratory Data Analysis}
Here you will explore the data. Naturally, the first plot you should make is the
data itself (Figure~\ref{fig:jj_data}). Here is how you insert figures into LaTex.

\begin{figure}[h!]
	\centering
	\includegraphics[width = 0.8\textwidth]{jj1.pdf}
	\caption{This is a figure. All figures should be captioned in the document.
	In the figure itself, axes should be properly labeled, and the font legible. }
	\label{fig:jj_data}
\end{figure}

Here you should point out any visible features, e.g. heteroscedasticity, seasonality, trend. Remember, this report is not to an audience familiar with time-series, so it's best to briefly introduce complicated terms as you use them (but do use them!).  

\section{Modeling a Deterministic Function of Time}

The next is to pursue stationarity. Before you can apply the ARIMA models learned in this class, you must make this data stationary. More plots should follow to justify why your transforms have worked. By the end of this section, you should have a transformed time series $Y_t$ that is stationary. 

It's possible to propose different models of time here, which you can use later for model comparison and diagnostics. 

\section{ARIMA Model Selection}
\label{sec:arima_model}
In this section, you derive an ARIMA (or SARIMA, MSARIMA, etc.) model, which you use in Section \ref{sec:results} to make predictions. 
%
On the way, you should compare different candidate models and do diagnostics for them (which can include different time functions from the previous section too). For example, perhaps you compare a parametric trend function with MA(1) noise to a second-differences with AR(1) noise model, etc.  
%
Go through the lecture materials and try your best to apply all the different approaches to find models, evaluate their accuracy, and make comparison between different models.
%
For example, there should certainly be ACF and PACF plots shown here!
%
Sometimes, it may be helpful to introduce subsections.

\subsection{Cross Validation Model Comparison}
For example, you may want to compare different models by how well the predict future values (as your predictions will also be evaluated in this final project).


	
	\begin{table}[h!]
	\centering
\begin{tabular}{lr}
	  Model & MSE \\ \hline
	  Second differences and MA(2) & 29.11 \\
	  Linear trend and ARMA(4,1) & 68.05 \\
		Linear trend and ARMA(1,2) & 99.07
	\end{tabular}
	\caption{These are the out-of-sample MSE's for our models of interest. There are many ways to do cross validation and there are many diagnostics you can look at. MSE is one example.}
	\label{tab:param_est}
\end{table}

At the end of this section you should decide on \textbf{one single model} you want to work with for your analysis. 

\section{Results}\label{sec:results}
Here, you will discuss the results of your model fit from
Section~\ref{sec:arima_model}. 
%
Math is very easy in Latex. Our ARIMA model is defined in equation (\ref{eq:first_arma_model}). See how I referenced the equation in LaTex?
\begin{align}
	(1 - \phi_1 B - \phi_2B^2) \nabla^2 X_t = (1 - \theta_1 B)Z_t
	\label{eq:first_arma_model}
\end{align}
Typically, some subsections will help with the organization of your material.
%
\subsection{Estimation of model parameters }
This might be a good time to introduce tables
in LaTex (Table~\ref{tab:param_est}).
All tables should be captioned, accordingly.

\begin{table}[h!]
	\centering
	\begin{tabular}{l|r}
	  Parameter & Estimate (s.e) \\ \hline
	  $\phi_1$& -0.29 (0.06) \\
	  $\phi_2$ & 0.068 (0.05) \\
		$\theta_1$ & -0.99(0.007)
	\end{tabular}
	\caption{These are our parameter estimates and corresponding standard errors
	for the ARIMA model in
	equation~\ref{eq:first_arma_model}. }
	\label{tab:param_est}
\end{table}

\newpage %in case you want a page break

\subsection{Prediction}
It's appropriate here to briefly discuss your predicted values and how they answer the problem/situation in the README file. Also, show them as new values on the time series plot of the actual data. It's helpful here to address uncertainty about your estimates too. 

Every group will be providing estimates of future values that are not in the dataset, and there will be one submission PER GROUP.
Only the instructor will have the actual values, and we will be evaluating your models. Please see the assignment .pdf for the particular format for submitting these values!






\end{document}
\grid
